\documentclass{article}
\usepackage[utf8]{inputenc}
\usepackage{amsmath}
\usepackage{environ}
\usepackage[T2A,T1]{fontenc}
\usepackage[russian]{babel}
\usepackage[letterpaper,top=2cm,bottom=2cm,left=3cm,right=3cm,marginparwidth=1.75cm]{geometry}

\begin{document}

\textbf{№ 10} 
\large

$$ \lim\limits_{x \to \frac{\pi}{2}} \frac{x}{x-\frac{\pi}{2}} \int_{\frac{\pi}{2}}^{x} \frac{\sin{t}}{t}dt $$ 

\begin{enumerate}
\item Рассмотрим табличный интеграл: интегральный синус
$$ \int_{0}^{x} \frac{\sin{t}}{t}dt 
= Si(x) $$

Или
$$ -\int_{x}^{\infty} \frac{\sin{t}}{t}dt 
= si(x) 
= Si(x) - \frac{\pi}{2}$$

Разложение в ряд Тейлора
$$ Si(x)
= x - \frac{x^3}{3\times3!} + \frac{x^5}{5\times5!} \ ...$$

Производная
$$ \left( Si(x) \right)'
= \frac{\sin{x}}{x} $$

А также предел интегрального синуса
$$ 
\begin{cases}
 \lim\limits_{x \to +\infty} &Si(x) = \frac{\pi}{2}\\
 \lim\limits_{x \to -\infty} &Si(x) = -\frac{\pi}{2}\\
 \lim\limits_{x \to +\infty} &si(x) = 0\\
 \lim\limits_{x \to -\infty} &si(x) = -\pi
\end{cases}
$$

\item Решаем первоначальное выражение
$$ \lim\limits_{x \to \frac{\pi}{2}} \frac{x}{x-\frac{\pi}{2}} \int_{\frac{\pi}{2}}^{x} \frac{\sin{t}}{t}dt
= \lim\limits_{x \to \frac{\pi}{2}} \frac{x}{x-\frac{\pi}{2}} \left( Si(x) - Si(\frac{\pi}{2}) \right)
= \frac{0}{0}
= $$

$$ = \lim\limits_{x \to \frac{\pi}{2}} \left( Si(x) - Si(\frac{\pi}{2}) \right) + x \left( \frac{\sin{x}}{x} - 0 \right)
= \left( Si(\frac{\pi}{2}) - Si(\frac{\pi}{2}) \right) \sin{\frac{\pi}{2}}
= 1 $$

\end{enumerate}
\end{document}