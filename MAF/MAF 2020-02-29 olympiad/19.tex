\documentclass{article}
\usepackage[utf8]{inputenc}
\usepackage{amsmath}
\usepackage{environ}
\usepackage[T2A,T1]{fontenc}
\usepackage[russian]{babel}
\usepackage[letterpaper,top=2cm,bottom=2cm,left=3cm,right=3cm,marginparwidth=1.75cm]{geometry}
\usepackage{xcolor}

\begin{document}

\textbf{№ 19} 
\large
\\
см. {\color{blue}MAF} 2018-06-16 olympiad № 12 \\
см. {\color{blue}MAF} 2020-02-29 olympiad № 19

$$ \lim\limits_{x \to +\infty} \left( \frac{x+a}{x-a} \right)^x = e; \ a = ? $$  

\begin{enumerate}
\item Решаем предел

$$ \lim\limits_{x \to +\infty} \left( \frac{x+a}{x-a} \right)^x
= \lim\limits_{x \to +\infty} e^{\ln{\left( \frac{x+a}{x-a} \right)^x}} $$

\item Решаем предел только в степени (на время уберем e)
$$ \lim\limits_{x \to +\infty} \ln{\left( \frac{x+a}{x-a} \right)^x} 
= \lim\limits_{x \to +\infty} \frac{\ln{\frac{x+a}{x-a}}}{\frac{1}{x}}  
= \frac{\infty}{\infty}
= \lim\limits_{x \to +\infty} \frac{ \frac{x-a}{x+a} \left( \frac{(x-a)-(x+a)}{(x-a)^2} \right) }{-\frac{1}{x^2}} 
= \lim\limits_{x \to +\infty} \frac{2ax^2}{(x+a)(x-a)} 
= $$

$$ = \lim\limits_{x \to +\infty} \frac{2ax^2}{x^2-a^2}  
= 2a \lim\limits_{x \to +\infty} \frac{x^2}{x^2 \left(1 - \frac{a^2}{x^2} \right)} 
= 2a $$

\item Подставляем решение в степень e и приравниваем к изначальному условию
$$ e^{2a} = e \ \Rightarrow \ a = \frac{1}{2} $$

\end{enumerate}
\end{document}