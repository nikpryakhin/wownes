\documentclass{article}
\usepackage[utf8]{inputenc}
\usepackage{amsmath}
\usepackage{environ}
\usepackage[T2A,T1]{fontenc}
\usepackage[russian]{babel}
\usepackage[letterpaper,top=2cm,bottom=2cm,left=3cm,right=3cm,marginparwidth=1.75cm]{geometry}

\begin{document}

\textbf{№ 5} 
\large

$$ \lim\limits_{x \to 0} \frac{\sqrt{ax + b} - \frac{2}{3}}{x}
= \frac{3}{4}, \ a = \ ? \ , \ b = \ ? $$  

$$ \lim\limits_{x \to 0} 
\frac{\left(\sqrt{ax + b} - \frac{2}{3}\right) \left(\sqrt{ax+b} + \frac{2}{3}\right)}
{x \left( \sqrt{ax+b} - \frac{2}{3}\right)} 
= \lim\limits_{x \to 0} \frac{ax + b - \frac{4}{9}}{x \left( \sqrt{ax+b} - \frac{2}{3}\right)} 
= \frac{3}{4} $$

$$ \lim\limits_{x \to 0} 4 \left( ax + b - \frac{4}{9} \right)
= \lim\limits_{x \to 0} 3x \left( \sqrt{ax+b} - \frac{2}{3}\right) $$

$$ 4 \left( b - \frac{4}{9} \right) 
= 0 
\
\Rightarrow 
\
b = \frac{4}{9} $$

Теперь подставляем b в изначальное уравнение

$$ \lim\limits_{x \to 0} \frac{\sqrt{ax + \frac{4}{9}} - \frac{2}{3}}{x}
= \frac{0}{0}
= \lim\limits_{x \to 0} \frac{a}{2\sqrt{ax+\frac{4}{9}}} $$

$$ \lim\limits_{x \to 0} \frac{a}{2\sqrt{ax+\frac{4}{9}}} 
= \frac{3}{4}
\
\Rightarrow 
\
\frac{a}{2\sqrt{\frac{4}{9}}}
= \frac{3}{4} 
\
\Rightarrow 
\
a = 1
$$

Итого $ a = 1, b = \frac{4}{9} $

\end{document}