\documentclass{article}
\usepackage[utf8]{inputenc}
\usepackage{amsmath}
\usepackage{environ}
\usepackage[T2A,T1]{fontenc}
\usepackage[russian]{babel}
\usepackage[letterpaper,top=2cm,bottom=2cm,left=3cm,right=3cm,marginparwidth=1.75cm]{geometry}

\begin{document}

\textbf{№ 12} 
\\
см. MAF 2020-06-06 olympiad № 5 \\
см. MAF 2018-03-03 olympiad № 12

$$ \lim\limits_{x \to 0} \frac{\sqrt{ax + b} - 2}{x}
= 1, \ a = \ ? \ , \ b = \ ? $$  

$$ \lim\limits_{x \to 0} \frac{\sqrt{ax + b} - 2}{x}
= \lim\limits_{x \to 0} \frac{\left( \sqrt{ax + b} - 2 \right) \left( \sqrt{ax + b} + 2 \right)}{x\left( \sqrt{ax + b} + 2 \right)} 
= \lim\limits_{x \to 0} \frac{ax+b-4}{x\left( \sqrt{ax + b} + 2 \right)} 
= 1 $$

$$ \Rightarrow ax+b-4 = x(\sqrt{ax+b}+2), \ x \to 0 $$
$$ \Rightarrow b-4 = 0  \Rightarrow b = 4$$

Подставляем $ b = 4 $ в первоначальное выражение и решаем его

$$ \lim\limits_{x \to 0} \frac{\sqrt{ax + 4} - 2}{x} 
= \frac{0}{0} 
= \lim\limits_{x \to 0} \frac{\frac{a}{2\sqrt{ax+4}} - 0}{1} 
= 1 $$

$$ \Rightarrow \frac{a}{2\sqrt{ax+4}} = 1, \ x \to 0 $$
$$ \Rightarrow a = 2\sqrt{ax+4} \ \Rightarrow a = 4 $$

Итого
$$ a = 4, \ b = 4 $$

\end{document}