\documentclass{article}
\usepackage[utf8]{inputenc}
\usepackage{amsmath}
\usepackage{environ}
\usepackage[T2A,T1]{fontenc}
\usepackage[russian]{babel}
\usepackage{hyperref}
\usepackage[letterpaper,top=2cm,bottom=2cm,left=3cm,right=3cm,marginparwidth=1.75cm]{geometry}

\begin{document}

\textbf{№ 29} 

\begingroup
\Large

$$ \lim\limits_{t\to 0} \left( \int_{0}^{1} (1+x)^t dx \right)^{\frac{1}{t}}
= $$

\begin{enumerate}
\item Решаем интеграл
$$ \int_{0}^{1} (1+x)^t dx
= \frac{(1+x)^{t+1}}{t+1}  \bigg\rvert_{0}^{1}
= \frac{2^{t+1}}{t+1} - \frac{1}{t+1}
= \frac{2^{t+1} - 1}{t+1}$$

\item Найдем предел
$$ \lim\limits_{t\to 0} \left( \frac{2^{t+1} - 1}{t+1} \right)^{\frac{1}{t}}
= 1^{\infty}
= e^{\lim\limits_{t\to 0} \frac{\ln \left( \frac{2^{t+1} - 1}{t+1} \right)}{t} }
= $$

\item Найдем только предел (без степени e)
$$ \lim\limits_{t\to 0} \frac{(t+1) (\ln{2}(t+1)2^{t+1} - 2^{t+1} + 2^{t+1} + 1)}{(2^{t+1} - 1)(t+1)^2}
= 2\ln{2}-1$$

\item Подставляем решение в степень e
$$ e^{2\ln{2}-1}
= e^{\ln{4} - 1} 
= \frac{4}{e}$$
\\
\\
II способ: после п.2 решать нужно по формуле $\lim\limits_{t\to \infty} \left( 1+\frac{1}{x} \right)^{x} = e$

\end{enumerate}
\endgroup
\end{document}