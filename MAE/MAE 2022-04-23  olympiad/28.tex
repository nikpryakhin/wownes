\documentclass{article}
\usepackage[utf8]{inputenc}
\usepackage{amsmath}
\usepackage{environ}
\usepackage[T2A,T1]{fontenc}
\usepackage[russian]{babel}
\usepackage[letterpaper,top=2cm,bottom=2cm,left=3cm,right=3cm,marginparwidth=1.75cm]{geometry}
\usepackage{xcolor}

\begin{document}
\textbf{№ 28} 
\\
\\
Небходимо построить график функции $\arctg{x}$ и $\frac{\pi}{4}x^2$, чтобы понимать, где ищем площадь (здесь должен быть нарисован график). Далее находим точки пересечений. Это 0 и 1
$$ S = \int_{0}^{1} \left( \arctg{x}-\frac{\pi}{4}x^2 \right) dx $$

\begin{enumerate}
\item Решаем первый интеграл
$$ \int \arctg{x}
= \begin{Vmatrix} \int f \,dg = fg - \int g \ df \end{Vmatrix} 
= \begin{Vmatrix} f = \arctg{x} &  g = x \\
                 df = \frac{1}{1+x^2}dx  & dg = dx \end{Vmatrix} 
= x\arctg{x} - \int \frac{x}{1+x^2} dx 
= $$

$$ = \begin{vmatrix} u = 1+x^2 \\
                    du = 2x \ dx \\
                    dx = \frac{du}{2x} \end{vmatrix} 
= x\arctg{x} - \frac{1}{2} \int \frac{du}{u}
= \left( x\arctg{x} - \frac{1}{2} \ln{(1+x^2)} \right) \bigg\vert_{0}^{1}
= \frac{\pi}{4} - \frac{1}{2} \ln{2} $$

\item Решаем второй интеграл
$$ \int \frac{\pi}{4}x^2 
= \left( \frac{\pi}{4} \frac{x^3}{3} \right) \bigg\vert_{0}^{1} 
= \frac{\pi}{12} $$

\item вычитаем из первого интеграла второй
$$ \frac{\pi}{4} - \frac{1}{2} \ln{2} - \frac{\pi}{12}
= \frac{\pi}{6} - \frac{1}{2} \ln{2} $$

\end{enumerate}
\end{document}